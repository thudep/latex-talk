\documentclass{beamer}
\usepackage{ctex, hyperref}
\usepackage[T1]{fontenc}

% other packages
\usepackage{latexsym,amsmath,xcolor,multicol,booktabs,calligra}
\usepackage{graphicx,pstricks,listings,stackengine}
\usepackage{capt-of}

\author{SphenHe}
\title{\LaTeX 使用简介}
\subtitle{能用就行!}
\institute{清华大学工程物理系}
\date{\today}
\usepackage{Tsinghua}

% defs
\def\cmd#1{\texttt{\color{red}\footnotesize $\backslash$#1}}
\def\env#1{\texttt{\color{blue}\footnotesize #1}}
\definecolor{deepblue}{rgb}{0,0,0.5}
\definecolor{deepred}{rgb}{0.6,0,0}
\definecolor{deepgreen}{rgb}{0,0.5,0}
\definecolor{halfgray}{gray}{0.55}

\lstset{
    basicstyle=\ttfamily\small,
    keywordstyle=\bfseries\color{deepblue},
    emphstyle=\ttfamily\color{deepred},    % Custom highlighting style
    stringstyle=\color{deepgreen},
    numbers=left,
    numberstyle=\small\color{halfgray},
    rulesepcolor=\color{red!20!green!20!blue!20},
    frame=shadowbox,
}


\begin{document}

\kaishu
\begin{frame}
    \titlepage
\end{frame}

\begin{frame}
    \tableofcontents[sectionstyle=show,subsectionstyle=show/shaded/hide,subsubsectionstyle=show/shaded/hide]
\end{frame}

\section{为什么是 \LaTeX}

\begin{frame}{Why \LaTeX}
    \begin{itemize}
        \item \LaTeX 广泛用于学术界,期刊会议论文模板
    \end{itemize}
    \begin{table}[h]
        \centering
        \begin{tabular}{c|c}
            Microsoft\textsuperscript{\textregistered}  Word & \LaTeX \\
            \hline
            文字处理工具 & 专业排版软件 \\
            容易上手,简单直观 & 容易上手 \\
            所见即所得 & 所见即所想,所想即所得 \\
            高级功能不易掌握 & 进阶难,但一般用不到 \\
            处理长文档需要丰富经验 & 和短文档处理基本无异 \\
            花费大量时间调格式 & 无需担心格式,专心作者内容 \\
            公式排版差强人意 & 尤其擅长公式排版 \\
            二进制格式,兼容性差 & 文本文件,易读、稳定 \\
            付费商业许可 & 自由免费使用 \\
        \end{tabular}
    \end{table}
\end{frame}

\begin{frame}{Why NOT Typst?}
    \begin{itemize}
        \item arXiv 不接受 Typst
    \end{itemize}
\end{frame}

\section{如何安装 \LaTeX}

\begin{frame}{如何配置 \LaTeX 环境?}
    \begin{itemize}
        \item 我认为,它不应该出现在我们的讲解中
        \item 实际上,我们也不一定需要配置 \LaTeX 环境
        \item 比如,使用 Overleaf 就是省心省事的选择
    \end{itemize}
    \begin{figure}[htbp]
        \centering
        \includegraphics[width=0.85\linewidth]{pic/overleaf.png}
        \caption{Overleaf}
        \label{fig:overleaf}
    \end{figure}
\end{frame}

\begin{frame}{如何配置 \LaTeX 环境?}
    \begin{itemize}
        \item 我认为,它不应该出现在我们的讲解中
        \item 实际上,我们也不一定需要配置 \LaTeX 环境
        \item 比如,使用 Overleaf 就是省心省事的选择
        \item 虽然……Overleaf is down, right before the NIPS deadline.
    \end{itemize}
    \begin{figure}[htbp]
        \centering
        \includegraphics[width=0.7\linewidth]{pic/overleaf_crash.jpg}
        \caption{Overleaf Crash}
        \label{fig:overleaf_crash}
    \end{figure}
\end{frame}

\begin{frame}{如何打开 overleaf?}
    \begin{itemize}
        \item Tsinghua Overleaf: \url{https://overleaf.tsinghua.edu.cn/}
        \item Overleaf: \url{https://www.overleaf.com/}
        \item 当然,如果你的电脑可以扫码,可以扫描以下二维码来进入 Tsinghua Overleaf
    \end{itemize}
    \begin{figure}[htbp]
        \centering
        \includegraphics[width=0.4\linewidth]{pic/tsinghua_overleaf.png}
        \caption{Tsinghua Overleaf}
        \label{fig:tsinghua_overleaf}
    \end{figure}
\end{frame}

\section{如何使用 \LaTeX}

\begin{frame}{初识 \LaTeX,让我们新建一个项目}
    \begin{itemize}
        \item 先认识一下这里面都是个啥
    \end{itemize}
    \begin{figure}[htbp]
        \centering
        \includegraphics[width=1.0\linewidth]{pic/newtolatex.png}
        \caption{New to \LaTeX}
        \label{fig:newtolatex}
    \end{figure}
\end{frame}

\begin{frame}{看起来你已经完全理解了!但是……好像不能使用中文?}
    \begin{itemize}
        \item 你需要使用 XeLaTeX 或 LuaLaTeX 编译
        \item 在 Overleaf 中,可以在 Menu $\to$ Settings $\to$ Compiler 中进行设置
        \item 你还需要使用 ctex 宏包来支持中文
    \end{itemize}
\end{frame}

\begin{frame}[fragile]{我们在小学二年级真的学过,文章要分段落}
    \begin{minipage}{0.5\linewidth}
        \begin{lstlisting}[language=TeX]
这是第一段内容。

这是第二段内容。
如果两段
之间没有空行,
则会被认为
是同一段内容。
        \end{lstlisting}
    \end{minipage}\hspace{1cm}
    \begin{minipage}{0.3\linewidth}
        这是第一个段落的内容。

        这是第二个段落的内容。
        如果两段之间没有空行,则会被认为是同一段内容。
    \end{minipage}
\end{frame}

\begin{frame}[fragile]{我只讲四点……这是第二个大点的第十个小点}
    \begin{minipage}{0.5\linewidth}
        \begin{lstlisting}[language=TeX]
\section{大点}
\subsection{小点}
\subsubsection{小小点}
% 没有更小的了
\section*{大点}
\subsection*{小点}
\subsubsection*{小小点}
        \end{lstlisting}
    \end{minipage}\hspace{1cm}
    \begin{minipage}{0.3\linewidth}
        相信并不难,请大家自行尝试完成
    \end{minipage}
\end{frame}

\begin{frame}[fragile]{虽然但是,答题还是要有条有理,分点作答}
    \begin{minipage}{0.5\linewidth}
\begin{lstlisting}[language=TeX]
\begin{itemize}
  \item A \item B
  \item C
  \begin{itemize}
    \item C-1
  \end{itemize}
\end{itemize}
\end{lstlisting}
    \end{minipage}\hspace{1cm}
    \begin{minipage}{0.3\linewidth}
        \begin{itemize}
            \item A
            \item B
            \item C
            \begin{itemize}
                \item C-1
            \end{itemize}
        \end{itemize}
    \end{minipage}
    \medskip
    \begin{minipage}{0.5\linewidth}
\begin{lstlisting}[language=TeX]
\begin{enumerate}
  \item 巨佬 \item 大佬
  \item 萌新
  \begin{itemize}
    \item[n+e] 瑟瑟发抖
  \end{itemize}
\end{enumerate}
\end{lstlisting}
    \end{minipage}\hspace{1cm}
    \begin{minipage}{0.3\linewidth}
        \begin{enumerate}
            \item 巨佬
            \item 大佬
            \item 萌新
            \begin{itemize}
                \item[n+e] 瑟瑟发抖
            \end{itemize}
        \end{enumerate}
    \end{minipage}
\end{frame}

\begin{frame}[fragile]{相信理工科的同学们一定会推导很多公式}
    \begin{columns}
        \begin{column}{.65\textwidth}
            \begin{lstlisting}[language=TeX]
$V = \frac{4}{3}\pi r^3$

\[
V = \frac{4}{3}\pi r^3
\]

% $$...$$ is the same as \[...\]

\begin{equation}
\label{eq:vsphere}
V = \frac{4}{3}\pi r^3
\end{equation}
            \end{lstlisting}
        \end{column}
        \begin{column}{.3\textwidth}
            $V = \frac{4}{3}\pi r^3$
            \[
                V = \frac{4}{3}\pi r^3
            \]
            \begin{equation}
                \label{eq:vsphere}
                V = \frac{4}{3}\pi r^3
            \end{equation}
        \end{column}
    \end{columns}
    \begin{itemize}
        \item 更多内容请看 \href{https://zh.wikipedia.org/wiki/Help:数学公式}{\color{purple}{这里}}
    \end{itemize}
\end{frame}

\begin{frame}[fragile]{公式推导完成后,我们就可以编写程序进行计算}
    \begin{lstlisting}[language=TeX]
\lstinputlisting[language=TeX]{code.tex}
    \end{lstlisting}
    \begin{lstlisting}[language=TeX]
\ begin{lstlisting}[language=TeX]
% Here is your code
% there should be no space after backslash
\ end{lstlisting}
    \end{lstlisting}
\end{frame}

\begin{frame}[fragile]{有人说,用数据说话,那就需要拿出表格}
    \begin{columns}
        \column{.6\textwidth}
\begin{lstlisting}[language=TeX]
\begin{table}[htbp]
    \caption{编号与含义}
    \label{tab:number}
    \centering
    \begin{tabular}{cl}
    \toprule
    编号 & 含义 \\
    \midrule
    1 & 4.0 \\
    2 & 3.7 \\
    \bottomrule
    \end{tabular}
\end{table}
公式~(\ref{eq:vsphere}) 的编号
与含义请参见
表~\ref{tab:number}。
\end{lstlisting}
        \column{.4\textwidth}
        \begin{table}[htpb]
            \centering
            \caption{编号与含义}
            \label{tab:number}
            \begin{tabular}{cl}\toprule
                编号 & 含义 \\\midrule
                1 & 4.0\\
                2 & 3.7\\\bottomrule
            \end{tabular}
        \end{table}
        \normalsize 公式~(\ref{eq:vsphere})的编号与含义请参见表~\ref{tab:number}。
        ~\\
        ~\\
        但是不如直接用网页版本的 \LaTeX 代码生成器 \url{https://www.tablesgenerator.com/latex_tables}
    \end{columns}
\end{frame}

\begin{frame}[fragile]{但是,表格还不够直观,我们还可以插入图片}
\begin{lstlisting}[language=TeX]
\begin{figure}[htbp]
    \centering
    \includegraphics[width=0.3\linewidth]{pic/Tsinghua_University_Logo.eps}
    \caption{Tsinghua University Logo}
    \label{fig:tsinghua_logo}
\end{figure}
\end{lstlisting}
\begin{figure}[htbp]
    \centering
    \includegraphics[width=0.3\linewidth]{pic/Tsinghua_University_Logo.eps}
    \caption{Tsinghua University Logo}
    \label{fig:tsinghua_logo}
\end{figure}
\end{frame}

\begin{frame}[fragile]{突然发现,图片还是要并排放才好看}
    \begin{itemize}
        \item 使用 minipage 宏包即可,也可以使用 subfigure 宏包
    \end{itemize}
\begin{lstlisting}[language=TeX]
\begin{minipage}{0.45\linewidth}
    % your code
\end{minipage}
\begin{minipage}{0.45\linewidth}
    % your code
\end{minipage}
\end{lstlisting}
    \begin{minipage}{0.45\linewidth}
        \centering
        \includegraphics[width=0.6\linewidth]{pic/Tsinghua_University_Logo.eps}
        \captionof{figure}{Tsinghua University Logo}
    \end{minipage}
    \begin{minipage}{0.45\linewidth}
        \centering
        \includegraphics[width=0.6\linewidth]{pic/Tsinghua_University_Logo.eps}
        \captionof{figure}{Tsinghua University Logo}
    \end{minipage}
\end{frame}

\begin{frame}[fragile]{有的时候,老师还会要求文字双栏排版}
    \begin{itemize}
        \item 使用 multicol 宏包即可
    \end{itemize}
\begin{lstlisting}[language=TeX]
\begin{multicols}{2}
    % your content
\end{multicols}
\end{lstlisting}
\begin{multicols}{2}
    这是双栏排版的内容。这是双栏排版的内容。这是双栏排版的内容。这是双栏排版的内容。这是双栏排版的内容。这是双栏排版的内容。这是双栏排版的内容。这是双栏排版的内容。这是双栏排版的内容。这是双栏排版的内容。这是双栏排版的内容。这是双栏排版的内容。这是双栏排版的内容。这是双栏排版的内容。这是双栏排版的内容。
\end{multicols}
\end{frame}

\begin{frame}[fragile]{以及,无论在什么时候,页码都是非常重要的}
    \begin{itemize}
        \item 这是 fancyhdr 宏包的功能
    \end{itemize}
\begin{lstlisting}[language=TeX]
\fancyhead[L]{SphenHe}
\fancyhead[C]{偏振光学实验}
\fancyfoot[C]{\thepage}
\end{lstlisting}
\end{frame}

\begin{frame}{或许,你还需要添加一份文章目录、图表索引、代码索引}
    \begin{itemize}
        \item 据说这是 tocloft 宏包的功能
        \item 文章目录使用 \cmd{tableofcontents} 命令生成
        \item 图表索引使用 \cmd{listoffigures} 和 \cmd{listoftables} 命令生成
        \item 代码索引使用 \cmd{listoflistings} 命令生成
    \end{itemize}
\end{frame}

\begin{frame}{最后,还应该有学术伦理,那还需要加入参考文献}
    \begin{itemize}
        \item 此处我们先跳过,一般建议从文献管理软件如 Zotero 直接导出 .bib 文件后,再进行文献引用。
    \end{itemize}
\end{frame}

\begin{frame}{顺带附上 \LaTeX 常用命令与环境}
    \begin{exampleblock}{命令}
        \centering
        \footnotesize
        \begin{tabular}{llll}
            \cmd{chapter} & \cmd{section} & \cmd{subsection} & \cmd{paragraph} \\
            章 & 节 & 小节 & 带题头段落 \\\hline
            \cmd{centering} & \cmd{emph} & \cmd{verb} & \cmd{url} \\
            居中对齐 & 强调 & 原样输出 & 超链接 \\\hline
            \cmd{footnote} & \cmd{item} & \cmd{caption} & \cmd{includegraphics} \\
            脚注 & 列表条目 & 标题 & 插入图片 \\\hline
            \cmd{label} & \cmd{cite} & \cmd{ref} \\
            标号 & 引用参考文献 & 引用图表公式等\\\hline
        \end{tabular}
    \end{exampleblock}
    \begin{exampleblock}{环境}
        \centering
        \footnotesize
        \begin{tabular}{lll}
            \env{table} & \env{figure} & \env{equation}\\
            表格 & 图片 & 公式 \\\hline
            \env{itemize} & \env{enumerate} & \env{description}\\
            无编号列表 & 编号列表 & 描述 \\\hline
        \end{tabular}
    \end{exampleblock}
\end{frame}

\section{\LaTeX 疑难杂症解决}

\begin{frame}[fragile]{听说,要编译好几次?}
    \begin{itemize}
        \item 一般来说,带目录、交叉引用的文档需要 xelatex 编译两次
        \item 如果有参考文献,一般需要在使用 xelatex 编译一次后,使用 bibtex 编译,再使用 xelatex 编译两次
        \item 如果嫌编译太慢,可以使用 draft 模式
        \item 当然,也可以尝试使用 xelatexmk 工具来自动化编译
    \end{itemize}
\end{frame}

\begin{frame}{以及,为什么不借助AI呢?}
    \begin{itemize}
        \item 我也想问,我觉得 AI 辅助写 \LaTeX 非常爽
    \end{itemize}
\end{frame}

\begin{frame}{讲到最后}
    \begin{itemize}
        \item 可以看见,\LaTeX 并不难上手,但是也不是那么好上手
        \item \LaTeX 编译很慢,虽然这并不是我今天才完成讲稿的理由
        \item 但是 \LaTeX 这一坨我们还是不得不吃 \pause
        \item 其实在日常生活中与记笔记的时候,Markdown 可能是更好的选择
        \item 但是今天我们看起来并不会讲 Markdown \pause
        \item Xinyu Xiang 同学将会为大家介绍 Typst
    \end{itemize}
\end{frame}

\begin{frame}
    \begin{center}
        {\Huge\calligra Thanks!}
    \end{center}
\end{frame}

\end{document}